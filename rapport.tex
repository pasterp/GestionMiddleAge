\documentclass[11pt,a4paper]{article}
\usepackage[utf8]{inputenc}
\usepackage[french,frenchb,francais]{babel}
\usepackage[T1]{fontenc}
\usepackage{amsmath}
\usepackage{amsfonts}
\usepackage{amssymb}
\author{\textsc{\textbf{Phélipot Pascal}}\\\textsc{Noro Geoffrey}\\\textsc{Ralijaona Tiona}\\\textsc{Rimoux Quentin}}
\title{Projet de BDD : Gestion au Moyen Age}
\date\today

\begin{document}
\maketitle
\pagestyle{plain}
\newpage\pagenumbering{Roman}
\tableofcontents

\newpage\pagenumbering{arabic}
\section{Introduction}
\subsection{L'idée}
Le projet est un jeu par navigateur orienté strategie en multijoueur. 
Chaque joueur dispose d'une base sur une carte. Il a la possibilité d'y construire des batiments, de gérer ses ressources et de lancer la recherche de nouvelles technologies.
Les ressources sont générées chaque minute pour chaque utilisateur et il faut améliorer ses batiments pour en produire plus. \\
Le but du jeu est d'attaquer les autres joueurs afin de gagner des ressources et d'améliorer sa base au maximum.


\subsection{Répartition des rôles}
\begin{itemize}
	\item a 
	\item b
	\item c
	\item d
\end{itemize}
\subsection{Mise en œuvre}
\subsection{Gestion de projet}

\newpage\section{Représentation des données}
\subsection{Modèle-Vue-Contrôleur}
\subsection{Encapsulation}
\subsection{Base de Données}

\newpage\section{Gestion des tours}
\subsection{Les possibilités}
\subsubsection{Ajax}
\subsubsection{Tache cron}
\subsection{Mise en œuvre}

\newpage\section{Interface Utilisateur}

\newpage\section{Conclusion}
\subsection{Avis personnels}
\subsubsection{Noro Geoffrey}
\subsubsection{Phélipot Pascal}
\subsubsection{Ralijaona Tiona}
\subsubsection{Rimoux Quentin}
\subsection{Avis Global}

\newpage\pagenumbering{Roman}
\section{Annexes}

\end{document}